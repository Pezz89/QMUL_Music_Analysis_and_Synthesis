\documentclass[titlepage]{scrartcl}
\usepackage{enumitem}
\usepackage[british]{babel}
\usepackage[style=apa, backend=biber]{biblatex}
\DeclareLanguageMapping{british}{british-apa}
\usepackage{url}
\usepackage{float}
\usepackage[labelformat=empty]{caption}
\restylefloat{table}
\usepackage{perpage}
\MakePerPage{footnote}
\usepackage{abstract}
\usepackage{graphicx}
% Create hyperlinks in bibliography
\usepackage{hyperref}
\usepackage{amsmath}

\usepackage[T1]{fontenc}
\usepackage[utf8]{inputenc}
\usepackage{blindtext}
\setkomafont{disposition}{\normalfont\bfseries}

\graphicspath{
    {./resources/},
}
\addbibresource{~/Documents/library.bib}

\newsavebox{\abstractbox}
\renewenvironment{abstract}
  {\begin{lrbox}{0}\begin{minipage}{\textwidth}
   \begin{center}\normalfont\sectfont\abstractname\end{center}\quotation}
  {\endquotation\end{minipage}\end{lrbox}%
   \global\setbox\abstractbox=\box0 }

\usepackage{etoolbox}
\makeatletter
\expandafter\patchcmd\csname\string\maketitle\endcsname
  {\vskip\z@\@plus3fill}
  {\vskip\z@\@plus2fill\box\abstractbox\vskip\z@\@plus1fill}
  {}{}
\makeatother

\DeclareCiteCommand{\citeyearpar}
    {}
    {\mkbibparens{\bibhyperref{\printdate}}}
    {\multicitedelim}
    {}

% MATLAB Code block stuff...
\usepackage{color}
\usepackage{listings}

\definecolor{dkgreen}{rgb}{0,0.6,0}
\definecolor{gray}{rgb}{0.5,0.5,0.5}

\lstset{language=Matlab,
   keywords={break,case,catch,continue,else,elseif,end,for,function,
      global,if,otherwise,persistent,return,switch,try,while},
   basicstyle=\ttfamily,
   keywordstyle=\color{blue},
   commentstyle=\color{gray},
   stringstyle=\color{dkgreen},
   numbers=left,
   numberstyle=\tiny\color{gray},
   stepnumber=1,
   numbersep=10pt,
   backgroundcolor=\color{white},
   tabsize=4,
   showspaces=false,
   showstringspaces=false}

\begin{document}
\title{ECS731P --- Music Analysis and Synthesis}
\subtitle{\LARGE{Assignment 2 Report}}
\author{Sam Perry --- EC16039}

\maketitle
\section{Task Solution}
The solution proposed is designed estimate the fundamental frequency of notes
played by a harpsichord, regardless of it's tuning. To achieve this, a number
of methods were employed for the extraction of fundamental frequencies and the
subsequent estimation of MIDI note frequencies for each of the harpsichord's
keys. The approach taken was based primarily around work by Dixon et.\ 
al~\citeyearpar{Dixon2012}.
This approach uses a quadratically interpolated FFT (QIFFT) for polyphonic
frequency estimation, combined with a selection of post-processing and analysis
techniques to produce an effective method for accurately estimating fundamental
frequencies ($f_0$) in harpsichord recordings.

\subsection{QIFFT based frequency extraction}
In order to estimate the frequencies of a harpsichord recording's notes
individually, the frequencies for notes played were extracted using a QIFFT
algorithm, combined with an adaptive threshold for peak picking.
This method is shown to be more accurate for harpsichord partial analysis than
techniques such as the phase-vocoder based instantaneous frequency analysis, as
discussed in~\parencite{Tidhar2010}. 
To apply this method, an adaptive threshold was first applied to the magnitude
spectrum. The threshold used combined moving mean and moving standard
deviation filters:
$$\lvert X(n,i) \rvert > \mu(n,i) + 0.5 \cdot \sigma(n,i).$$
By removing all frequencies that lied below this threshold, local maxima formed
by harmonic content could be separated from noise related peaks effectively. A
static threshold at -25dB less than the global maximum value was set to remove
any low level noise:
$$\lvert X(n,i) \rvert > 10^{-2.5}\max_{u,v}\{\lvert X(u,v)\}.$$
At this point local maxima could be picked for all points above the threshold
to generate a set of harmonic peaks. This is shown in figure

\subsection{Quadratic Interpolation}
In order to gain a more accurate estimation than is possible within the limits
of the FFT's bin frequencies, peaks were interpolated by fitting a quadratic
parabola:
$$\delta = \frac{a_{p-1}-a_{p+1}}{2(a_{p-1}-2a_p+a_{p+1}})$$.
By taking the estimated peak and it's two neighbouring bin magnitudes, a
more precise prediction is produced by fitting a parabolic function to these
samples and calculating it's maxima. This helps to produce the high levels of
accuracy needed for the distinction of subtle deviations in fundamental
frequency.

\subsection{Harmonic multiples filtering}\label{HMF}
As the QIFFT extraction method has no way of discerning between a fundamental
partial and a harmonic, a method for removing some of the masking harmonics is
presented by Dixon et. al~\citeyearpar{Dixon2012}. Harmonics are removed by
iterating over harmonics ordered by their magnitude. Ratios are calculated in
terms of cents between the current harmonic's multiples and the following
harmonics frequency. If any harmonics are found to be within a set range (50
cents) of a multiple of the $f_0$, it is removed. This helps reduce the
interference of harmonics, at the cost of potential loss of future $f_0$.

\subsection{Estimation of note frequencies}
Having estimated $f_0$ values for all notes in a recording, the final stage is
to classify partials as MIDI notes to produce the final vector. This was
achieved by initially using a base nominal pitch for the harpsichord (set to
415Hz). Partials were selected from a range of half a semitone (in a similar
way to that of section~\ref{HMF} above or below this value. The median was then
calculated to produce an estimate for this note. From this, estimates for all
subsequent notes could be calculated in a similar manner, using the formula
described in~\parencite{Scarff2017} to estimate the base pitch.

\section{Evaluation}
It was decided that, given limited time, focus would be placed on implementing
a robust proven algorithm over the design and extensive testing of a more
experimental approach. This approach was chosen for it's proven accurate
results and as such, is expected to perform to a reasonable standard, providing
it's implementation is correct. The algorithm has been tested using the two
sample audio files provided and produces a vector of frequency values that do
not appear to be unexpected. Individual components such as parabolic
interpolation appear to perform as expected and have been tested on small
segments of audio files to verify the improvement in frequency estimation on
monophonic notes. Harmonic removal was also tested through the use of
fabricated peak value vectors to check that the correct values were removed as
expected.\\

However, it is clear that further, more rigorous testing is required to verify
that the algorithm is working as expected. The use of a larger test database
would confirm the generalization of the algorithm, as well as potentially
highlighting edge-cases that may have been missed. Most importantly the
frequency values that have been computed should be inspected in greater detail.
More thorough testing through use of known values would allow for the
quantification of results that is currently lacking.\\

Comparison with other methods such as the phase-vocoder implementation would
provide an alternate system for comparison of results. It is also noted that
the harmonicity estimation was not implemented. This would have provided a more
interesting metric for analysing the confidence of estimations.

\section{Figures}
\begin{figure}[H]
    \caption{Adaptive threshold (green) applied to magnitude spectrum to find
    peaks (orange)}
    \makebox[\textwidth]{\includegraphics[width=1.1\textwidth]{MeanStdPeaks}}
    \label{MeanStdPeaks}
\end{figure}
\begin{figure}[H]
    \caption{Picking local maxima across filtered magnitude spectrum}
    \makebox[\textwidth]{\includegraphics[width=1.1\textwidth]{PeakPicking}}
    \label{PeakPicking}
\end{figure}

\section{Notes}
Unfortunately Latex will not print the URL for the Scraff reference properly.
The url is:
\href{https://www.midikits.net/midi_analyser/midi_note_frequency.htm}{https://www.midikits.net/midi\_analyser/midi\_note\_frequency.htm}
\printbibliography
\end{document}
