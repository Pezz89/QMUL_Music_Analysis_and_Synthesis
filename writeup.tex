\documentclass[titlepage]{scrartcl}
\usepackage{enumitem}
\usepackage[british]{babel}
\usepackage[style=apa, backend=biber]{biblatex}
\DeclareLanguageMapping{british}{british-apa}
\usepackage{url}
\usepackage{float}
\usepackage[labelformat=empty]{caption}
\restylefloat{table}
\usepackage{perpage}
\MakePerPage{footnote}
\usepackage{abstract}
\usepackage{graphicx}
% Create hyperlinks in bibliography
\usepackage{hyperref}
\usepackage{amsmath}

\usepackage[T1]{fontenc}
\usepackage[utf8]{inputenc}
\usepackage{blindtext}
\setkomafont{disposition}{\normalfont\bfseries}

\graphicspath{
    {./resources/},
}
\addbibresource{~/Documents/library.bib}

\newsavebox{\abstractbox}
\renewenvironment{abstract}
  {\begin{lrbox}{0}\begin{minipage}{\textwidth}
   \begin{center}\normalfont\sectfont\abstractname\end{center}\quotation}
  {\endquotation\end{minipage}\end{lrbox}%
   \global\setbox\abstractbox=\box0 }

\usepackage{etoolbox}
\makeatletter
\expandafter\patchcmd\csname\string\maketitle\endcsname
  {\vskip\z@\@plus3fill}
  {\vskip\z@\@plus2fill\box\abstractbox\vskip\z@\@plus1fill}
  {}{}
\makeatother

\DeclareCiteCommand{\citeyearpar}
    {}
    {\mkbibparens{\bibhyperref{\printdate}}}
    {\multicitedelim}
    {}

% MATLAB Code block stuff...
\usepackage{color}
\usepackage{listings}

\definecolor{dkgreen}{rgb}{0,0.6,0}
\definecolor{gray}{rgb}{0.5,0.5,0.5}

\lstset{language=Matlab,
   keywords={break,case,catch,continue,else,elseif,end,for,function,
      global,if,otherwise,persistent,return,switch,try,while},
   basicstyle=\ttfamily,
   keywordstyle=\color{blue},
   commentstyle=\color{gray},
   stringstyle=\color{dkgreen},
   numbers=left,
   numberstyle=\tiny\color{gray},
   stepnumber=1,
   numbersep=10pt,
   backgroundcolor=\color{white},
   tabsize=4,
   showspaces=false,
   showstringspaces=false}

\begin{document}
\title{ECS731P - Music Analysis and Synthesis}
\subtitle{\LARGE{Assignment 2 Report}}
\author{Sam Perry - ec16039}

\maketitle
\section{Task Solution}
The solution proposed is designed estimate the fundamental frequency of notes
played by a harpsichord, regardless of it's tuning. To achieve this, a number
of methods were employed for the extraction of fundamental frequencies and the
subsequent estimation of MIDI note frequencies for each of the harpsichord's
keys. The approach taken was based primarily around work by Dixon et.\ 
al~\citeyearpar{Dixon2012}.
This approach uses a quadratically interpolated FFT (QIFFT) for polyphonic
frequency estimation, combined with a selection of post-processing and analysis
techniques to produce an effective method for accurately estimating fundamental
frequencies ($f_0$) in harpsichord recordings.

\subsection{QIFFT based frequency extraction}
In order to estimate the frequencies of a harpsichord recording's notes
individually, the frequencies for notes played were extracted using a QIFFT
algorithm, combined with an adaptive threshold for peak picking.
This method is shown to be more accurate for harpsichord partial analysis than
techniques such as the phase-vocoder based instantaneous frequency analysis, as
discussed in~\parencite{Tidhar2010}. 
To apply this method, an adaptive threshold was first applied to the magnitude
spectrum. The threshold used combined moving mean and moving standard
deviation filters:
$$\lvert X(n,i) \rvert > \mu(n,i) + 0.5 \cdot \sigma(n,i).$$
By removing all frequencies that lied below this threshold, local maxima formed
by harmonic content could be separated from noise related peaks effectively. A
static threshold at -25dB less than the global maximum value was set to remove
any low level noise:
$$\lvert X(n,i) \rvert > 10^{-2.5}\max_{u,v}\{\lvert X(u,v)\}.$$
At this point local maxima could be picked for all points above the threshold
to generate a set of harmonic peaks. This is shown in figure

\subsection{Quadratic Interpolation}
In order to gain a more accurate estimation than is possible within the limits
of the FFT's bin frequencies, peaks were interpolated by fitting a quadratic
parabola:
$$\delta = \frac{a_{p-1}-a_{p+1}}{2(a_{p-1}-2a_p+a_{p+1}})$$.
By taking the estimated peak and it's two neighbouring bin magnitudes, a
more precise prediction is produced by fitting a parabolic function to these
samples and calculating it's maxima. This helps to produce the high levels of
accuracy needed for the distinction of subtle deviations in fundamental
frequency.

\subsection{Harmonic multiples filtering}
As the QIFFT extraction method has no was of discerning between a fundamental
partial and a harmonic, a method for removing some of the masking harmonics is
presented by Dixon et. al~\citeyearpar{Dixon2012}. By calculating the 

\section{Evaluation}
Program assumes that peaks will be near
why not use YIN
\subsection{Other known bugs}
\begin{itemize}
    \item a
\end{itemize}

\subsection{Possible improvements}
\begin{itemize}
    \item a
\end{itemize}

\section{Figures}
\begin{figure}[H]
    \caption{Adaptive threshold (green) applied to magnitude spectrum to find
    peaks (orange)}
    \makebox[\textwidth]{\includegraphics[width=1.1\textwidth]{MeanStdPeaks}}
    \label{MeanStdPeaks}
\end{figure}
\begin{figure}[H]
    \caption{Picking local maxima across filtered magnitude spectrum}
    \makebox[\textwidth]{\includegraphics[width=1.1\textwidth]{PeakPicking}}
    \label{PeakPicking}
\end{figure}
\printbibliography
\end{document}
